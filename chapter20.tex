\section{Einfache Populationsmodelle}

\begin{karte}{Malthusisches Wachstumsgesetz}
    Sei \(p(t) \) die Population einer Spezies zur Zeit \(t\) 
    und \(r(t,p)\) die Differenz zwischen Geburts- und Sterberate. 
    Ist die Spezies isoliert, so ist 
    \[ \frac{dp}{dt} = r(t,p)p(t). \]
    Einfachste Annahme: \(r(t,p) = a\) ist konstant. 
    \[ \frac{dp}{dt} = ap. \]
    Lösung ist gegeben durch 
    \[ p(t) = p(t_0) \exp(a(t_0-t)) \;\forall t\in \R. \]
\end{karte}

\begin{karte}{Logistische Gleichung}
    Ist die Population sehr groß, wird sie nicht ohne weiteres 
    exponentiell weiter anwachsen können. \(\Rightarrow -bp^2\) mit 
    \(b>0\) konstant.
    \[ \frac{dp}{dt} = ap - bp^2, p(t_0) = p_0 \]
    ist eine separable Gleichung. Es ergeben sich drei Fälle:
    \begin{enumerate}
        \item \( p_0 = a/b \), so ist \(p(t) = a/b\) eine Lösung (Gleichgewicht)
        \item Ist \(p_0 = 0\), so ist \(p(t) = 0\) eine Lösung.
        \item Sei \(p_0 \neq 0, p_0 \neq a/b\). 
    \end{enumerate}
    \[ \Rightarrow p(t) = \frac{a p_0}{b p_0 + (a-bp_0) e^{-a(t-t_0)}}, \]
    wenn \(p_0 \neq 0\) und \(p_0 \neq \frac{a}{b}\) ist.
    Grenzpopulation ist \(a/b\).\\
    Ist \(0<p_0<a/b\), so ist \(t\mapsto p(t)\) wachsend. 
    Ist \(p_0 > a/b\), so ist \(t\mapsto p(t)\) fallend.
\end{karte}