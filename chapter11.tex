\section{Einf. zshgd. Gebiete}

\subsection{Holomorph zshgd.}

\begin{karte}{Holomorph einfach zusammenhängend}
    Ein Gebiet \( U \subset \C\) (d. h. offen und zusammenhängend) heißt 
    holomorph einfach zusammenhängend, falls für jede geschlossene Kurve \(\gamma\)
    in \(U\) und jede analytische Funktion \(\abb{f}{U}{\C}\) auch 
    \[ \int_\gamma f(z) \dx{z} = 0 \]
    ist.\\
    Es gilt 
    holomorph einfach zusammenhängend \(\Leftrightarrow\) (topologisch) einfach zusammenhängend.
\end{karte}

\begin{karte}{Logarithmus und Wurzel}
    Sei \(G\) ein holomorph einfach zusammenhängendes Gebiet und \(\abb{f}{G}{\C}\) 
    holomorph ohne Nullstellen in \(G\). Dann gilt
    \begin{enumerate}
        \item Es gibt eine analytische Funktion 
        \[ \abb{h}{G}{\C} \text{ mit } e^h = f. \]
        \item Für alle \(n = 2,3,4,\ldots \) existieren analytische Funktionen 
        \(\abb{g_n}{G}{\C}\) mit 
        \[ (g_n)^n = f. \]
    \end{enumerate}
    \(h\) ist nur bis auf ganzzahlige Vielfache von \(2\pi i\) bestimmt und \(g_n\) eindeutig 
    bis auf \(n\)-te Einheitswurzeln.\\
    Wir schreiben auch \(\log f\) oder \(f^{1/n}\).
\end{karte}

\begin{karte}{Konforme äquivalenz von holomorph und einfach zusammenhängend}
    Seien \(U\) ein holomorph einfach zusammenhängendes Gebiet und 
    \(\tilde{U}\) ein weiteres Gebiet, für die es eine analytische Bijektion 
    \(\abb{g}{U}{\tilde{U}}\) gibt, für die auch \(g^{-1}\) analytisch ist. 
    Dann ist auch \(\tilde{U}\) holomorph einfach zusammenhängend.
\end{karte}

\begin{karte}{Annulus und Sektor}
    Gegeben \(0 < r < R < \infty\) sei 
    \[ A_{r,R} := \set{ z\in \C : r < \abs{z} < R } \]
    der Ring (Annulus) und für \(\alpha < \beta\) reell mit \(\abs{\beta - \alpha} \leq 2 \pi\) setze 
    \[ S_{\alpha, \beta} := \set{r e^{i\theta}: r > 0, \alpha < \theta < \beta} \]
    (Sektor).
\end{karte}

\begin{karte}{Annulus Sektor zusammenhängend}
    Sei \(0 \leq r < R < \infty \) und \(\alpha, \beta \in \R, \alpha < \beta\) 
    und \( \abs{\beta - \alpha} \leq 2\pi \). Dann ist \(A_{r,R} \cap S_{\alpha, \beta}\) 
    holomorph einfach zusammenhängend.\\
    Sei zusätzlich \( \cos\left( \frac{\beta - \alpha}{2} \right) > \frac{r}{R} \). 
    Dann ist \( A_{r,R} \cap S_{\alpha, \beta} \) sternförmig (und somit holomorph einfach zusammenhängend).
\end{karte}

\subsection{Homotopieinv. d. Linienint.}

\begin{karte}{Homotope Kurven}
    Zwei Kurven \(\abb{\gamma_0, \gamma_1}{[0,1]}{U}\), \(U\) Gebiet 
    mit \(\gamma_0(0) = \gamma_1(0), \gamma_0(1) = \gamma_1(1)\) sind 
    homotop, falls es eine stetige Funktion \(\abb{\Gamma}{[0,1]^2}{U}\) 
    gibt, sodass 
    \[ \Gamma(t,0) = \gamma_0(t) \text{ und } \Gamma(t,1) = \gamma_1(t) 
    \text{ für alle } t\in [0,1]. \]
    Sind \(\gamma_0, \gamma_1\) zwei \(PC^1\)-Kurven, welche homotop sind 
    und \(\abb{f}{U}{\C}\) analytisch, dann gilt: 
    \[ \int_{\gamma_0} f(z) \dx{z} = \int_{\gamma_1} f(z) \dx{z}. \]
\end{karte}

\begin{karte}{Homotopie aus homotopen Kurven}
    Sind \(\abb{\gamma_0, \gamma_1}{[0,1]}{U}\) \(PC^1\)-Kurven mit gleichen 
    Anfangs- und Endpunkten, welche homotop sind, so existiert eine Homotopie 
    \(\abb{\tilde{\Gamma}}{[0,1]^2}{U}\) so, dass für alle \(s \in [0,1]\) 
    die Funktion \(\tilde{\Gamma}(\cdot, s)\) \(PC^1\) ist.
\end{karte}

\begin{karte}{Lokal konstante Kurvenintegrale aus Homotopien}
    Sind \(\abb{\gamma_0, \gamma_1}{[0,1]}{U}\) zwei homotope \(PC^1\)-Kurven 
    mit \(\gamma_0(0) = \gamma_1(0), \gamma_0(1) = \gamma_1(1)\) und 
    \( \abb{\tilde{\Gamma}}{[0,1]^2}{U} \) die Homotopie aus dem Lemma davor 
    und \(\abb{f}{U}{\C}\) analytisch, so ist 
    \[ s \mapsto h(s) := \int_0^1 f(\tilde{\Gamma}(t,s)) \partial_t \tilde{\Gamma}(t,s) \dx{t} \]
    lokal konstant.
\end{karte}

\begin{karte}{Homotopieinvarianz des Linienintegrals}
    Sei \(U \subset \C\) offen, \(\abb{f}{U}{\C}\) analytisch 
    und \(\gamma_0, \gamma_1\) zwei Kurven in \(U\) die homotop sind und 
    dieselben Anfangs- und Endpunkte besitzen. Dann ist 
    \[ \int_{\gamma_0} f(z) \dx{z} = \int_{\gamma_1} f(z) \dx{z}. \]
\end{karte}

\begin{karte}{Einfach zusammenhängend}
    Ein Gebiet \(U \subset \C\) heißt einfach zusammenhängend, falls 
    für jede geschlossene Kurve \(\abb{\gamma}{[0,1]}{U}\) 
    eine Homotopie \(\abb{\Gamma}{[0,1]^2}{U}\) existiert, sodass 
    \[ \Gamma(t,1) = \gamma(0) \text{ für alle } t\in [0,1]. \]
    Beispiele: 
    \begin{enumerate}
        \item Konvexe und sternförmige Gebiete
        \item Nicht einfach zusammenhängend: Annulus
    \end{enumerate}
    Jedes einfach zusammenhängende Gebiet \(U \subset \C\) ist 
    holomorph einfach zusammenhängend.
\end{karte}

\subsection{Logarithmus}

\begin{karte}{Analytischer Zweig des Logarithmus}
    Eine Funktion \(\abb{h}{U}{\C}\) mit 
    \(\exp(h(z)) = z\) heißt analytischer Zweig des Logarithmus. 
    Dies ist der Fall, wenn 
    \begin{enumerate}
        \item \(h\) analytisch in \(U\) ist.
        \item \(h\) eine Inverse von \(\exp\) auf \(U\) ist.
    \end{enumerate}
    Ist \(h\) ein analytischer Zweig des Logarithmus, so ist dies auch 
    \[ g(z) := h(z) + 2\pi i k \]
    für jedes feste \(k\in \Z\).
\end{karte}

\begin{karte}{Beispiel für analytischen Zweig des Logarithmus}
    Sei \(U \subset \C\) ein einfach zusammenhängendes Gebiet 
    mit \(0 \notin U\). Für jede Wahl von \(z_0 \in U\) und 
    \(h_0 \in \C\) ist 
    \[ \log(z) := \int_{z_0}^z \frac{d \zeta}{\zeta}  + h_0 = \int_{\gamma_{z_0, z}} \frac{d\zeta}{\zeta} + h_0 \]
    ein analytischer Zweig des Logarithmus. Hier ist \(\gamma_{z_0, z}\) 
    eine beliebige Kurve von \(z_0\) nach \(z\) in \(U\).
\end{karte}