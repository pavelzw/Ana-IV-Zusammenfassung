\section{Funktionen von \(z\)}

\subsection{Analytische Polynome}

\begin{karte}{Allgemeines Polynom}
    Ein allgemeines Polynom auf \(\C = \R^2\) 
    ist eine endliche Linearkombination von Monomen, d. h. 
    \[ P(x,y) = \sum_{j=1}^m \sum_{k=1}^n a_{jk} x^j y^k. \]
\end{karte}

\begin{karte}{Analytisches Polynom}
    Ein Polynom \(P(x,y)\) heißt analytisch, falls es komplexe Zahlen 
    \( \alpha_0, \ldots, \alpha_N \) gibt, sodass 
    \[ P(x,y) = \alpha_0 + \alpha_1 (x+iy) + \cdots + \alpha_N (x+iy)^N. \]
    Wir sagen dann, dass \(P\) ein Polynom in \(z\) ist und schreiben 
    \[ P(z) = \alpha_0 + \alpha_1 z + \cdots + \alpha_N z^N. \]
\end{karte}

\begin{karte}{Partielle Ableitungen}
    Sei \( f(x,y) = u(x,y) + iv(x,y) \), \(u,v\) reellwertige Funktionen. 
    Die partiellen Ableitungen \(f_x\) und \(f_y\) sind definiert als 
    \[ f_x := u_x + i v_x, \qquad f_y := u_y + i v_y, \]
    sofern \(u\) und \(v\) partielle Ableitungen haben. 
\end{karte}

\begin{karte}{Cauchy-Riemann Differentialgleichungen}
    Ein Polynom \(P\) ist genau dann analytisch, wenn \(P_y = i P_x\).\\
    Oder äquivalent: 
    \[ u_x = v_y \text{ und } u_y = -v_x. \]
\end{karte}

\subsection{Komplex differenzierbar}

\begin{karte}{Komplex differenzierbar}
    Eine komplexwertige Funktion \(\abb{f}{U}{\C}, U\) Umgebung um \(z\), 
    ist (komplex) differenzierbar in \(z\in U\), falls 
    \[ \limesx{h}{0} \frac{f(z+h) - f(z)}{h} \] existiert. 
    In diesem Fall setzen wir 
    \[ f'(z) := \limesx{h}{0} \frac{f(z+h) - f(z)}{h}. \]
    Wir nennen \(f\) differenzierbar, falls \(f\) in jedem Punkt \(z\in U\) (komplex) 
    differenzierbar ist.
\end{karte}

\begin{karte}{Produktregel, Kettenregel}
    Sind \(f,g\) komplex differenzierbar in \(z\), so gilt dies auch für \(f+g\) und \(fg\) 
    und ist \(g(z) \neq 0\), so ist \(\frac{f}{g}\) differenzierbar. Es gilt: 
    \begin{enumerate}
        \item \((f+g)'(z) = f'(z) + g'(z)\)
        \item \((fg)'(z) = f'(z)g(z) + f(z)g'(z)\)
        \item \( (\frac{f}{g})'(z) = \frac{f'(z) g(z) - f(z) g'(z)}{g(z)^2} \)
    \end{enumerate}
\end{karte}

\begin{karte}{Analytische Polynome komplex differenzierbar}
    Ist \(P(z) = a_0 + a_1 z + \cdots + a_N z^N\) ein analytisches Polynom, dann ist \(P\) komplex differenzierbar 
    in jedem Punkt \(z\in\C\) und es gilt 
    \[ P'(z) = \alpha_1 + 2\alpha_2 z + 3\alpha_3 z^2 + \cdots + N \alpha_N z^{N-1}. \]
\end{karte}

\subsection{Potenzreihen}

\begin{karte}{Potenzreihe}
    Eine Potenzreihe in \(z\) ist eine unendliche Reihe der Form 
    \[ \sum_{n=0}^\infty a_n z^n \text{ oder } \sum_{n=0}^\infty a_n (z - z_0)^n. \]
\end{karte}

\begin{karte}{Konvergenzradius}
    Angenommen \( L = \limsup_{n\rightarrow\infty} \abs{a_n}^{1/n} \). Dann gilt: 
    \begin{enumerate}
        \item Ist \(L = 0\), so konvergiert \( \sum_{n=0}^\infty a_n z^n \) für alle \(z\in \C\).
        \item Ist \( L = \infty \), so konvergiert \( \sum_{n=0}^\infty a_n z^n \) für alle \(z\in \C\).
        \item Ist \( 0 < L < \infty \), setze \(R := \frac{1}{L}\). Dann konvergiert \( \sum_{n=0}^\infty a_n z^n \) für alle \(\abs{z} < R\) 
        und divergiert für alle \(\abs{z} > R\).
    \end{enumerate}
    \(R\) ist der Konvergenzradius. Falls \(R > 0\) ist, gilt, dass für alle \( 0<\tilde{R}<R \) die Potenzreihe 
    \(\sum_{n=0}^\infty a_n z^n\) gleichmäßig für alle \(\abs{z} < \tilde{R}\) konvergiert.
\end{karte}

\begin{karte}{Ableitung Potenzreihe}
    Angenommen \(f(z) := \sum_{n=0}^\infty a_n z^n\) konvergiert für \(\abs{z} < R\). 
    Dann ist \(f\) komplex differenzierbar und 
    \[ f'(z) = \sum_{n=1}^\infty n a_n z^{n-1} \text{ für alle } \abs{z} < R. \]
    Jede Potenzreihe ist innerhalb ihres Konvergenzradius unendlich oft differenzierbar. 
    Hat \( f(z) = \sum_{n=0}^\infty a_n z^n \) einen Konvergenzradius \(R > 0\), so gilt 
    \[ a_n = \frac{f^{(n)}(0)}{n!}. \]
\end{karte}

\subsection{Eindeutigkeit Potenzreihen}

\begin{karte}{Eindeutigkeit von Potenzreihen}
    Angenommen \(f(z) = \sum_{n=0}^\infty a_n z^n\) konvergiert für ein 
    \(z \neq 0\) und \(f(z_n) = 0\) für eine Nullfolge \((z_n)_n\) mit \(z_n \neq 0\) 
    für alle \(n\in\N\). Dann gilt 
    \[ a_n = 0 \forall n\in\N_0. \]
\end{karte}

\begin{karte}{Potenzreihe mit Häufungspunkt \(0\)}
    Ist die Potenzreihe \(\sum_{n=0}^\infty a_n z^n\) Null für alle \(z\) in einer Menge \(S\), 
    die den Häufungspunkt Null hat, so ist \(a_n = 0\) für alle \(n\in\N_0\).
\end{karte}

\begin{karte}{Gleichheit von Potenzreihen}
    Konvergieren \(\sum_{n=0}^\infty a_n z^n\) und \(\sum_{n=0}^\infty b_n z^n\) und sind diese gleich für alle \(z\) in 
    einer Menge \(S\) mit Häufungspunkt Null, so ist \(a_n = b_n\) für alle \(n\in\N_0\).
\end{karte}