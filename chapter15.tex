\section{Residuensatz}

\begin{karte}{Residuum}
    Ist \(f(z) = \sum_{n=-\infty}^\infty a_n (z-z_0)^n \) die Laurententwicklung 
    von \(f\)in einer punktierten Umgebung von \(z_0\), so heißt \(a_{-1}\) 
    das Residuum von \(f\) in \(z_0\):
    \[ \Res(f,z_0) := a_{-1}. \]
    \begin{itemize}
        \item Hat \(f\) einen einfachen Pol in 
        \(z_0\), \(f(z) = \frac{A(z)}{B(z)}\), \(A,B\) analytisch in \(z_0\)
        und \(B\) hat eine einfache Nullstelle in \(z_0\)
        \[ \Rightarrow a_{-1} = \limesx{z}{z_0} (z-z_0) f(z) = \frac{A(z_0)}{B'(z_0)}. \]
        \item \(f\) hat einen Pol der Ordnung \(k\) in \(z_0\)
        \[ \Rightarrow \Res(f,z_0) = \frac{1}{(k-1)!} \frac{d^{k-1}}{dz^{k-1}} [(z-z_0)^k f(z)]|_{z=z_0}. \]
        \item In den meisten Fällen von Polen sehr hoher Ordnung oder einer 
        wesentlichen Singularität ist es oft am einfachsten, \(a_{-1}\) aus der 
        Laurententwicklung abzulesen.
    \end{itemize}
\end{karte}

\begin{karte}{Meromorph}
    Sei \(U\subset \C\) ein Gebiet. \(f\) heißt meromorph, 
    falls \(f\) analytisch in \(U\) bis auf endlich viele isolierte Pole ist.
    D. h. es existieren \(z_1, \ldots, z_N \in U\), sodass 
    \(\abb{f}{U\setminus \set{z_1, \ldots, z_N}}{\C}\) analytisch ist.
\end{karte}

\begin{karte}{Cauchyscher Residuensatz}
    Angenommen \( f \) ist meromorph in einem einfach zusammenhängenden Gebiet 
    \(U\) mit isolierten Singularitäten \(z_1, \ldots,z_N\in U\) und \(\gamma\) 
    eine geschlossene Kurve in \(U\) und \(z_j \notin \Bild(\gamma)\) für alle 
    \(j=1,\ldots, N\). Dann gilt 
    \[ \int_\gamma f(z) \dx{z} = 2\pi i \sum_{k=1}^N n(\gamma, z_k) \Res(f,z_k). \]
\end{karte}

\begin{karte}{Cauchyscher Residuensatz bei regulären Kurven}
    Angenommen \(f\) ist meromorph in einem einfach zusammenhängenden Gebiet \(U\)
    mit isolierten Singularitäten und \(\gamma\) eine reguläre geschlossene 
    Kurve in einer Teilmenge von \(U\), in der \(f\) analytisch ist. Dann ist 
    \[ \int_\gamma f(z) \dx{z} = 2\pi i \sum_k \Res(f,z_k), \]
    wobei die Summe über diejenigen \(k\) ist, für die die Singularität von 
    \(f\) im Inneren von \(\gamma\) ist.
\end{karte}