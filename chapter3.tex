\section{Analytische Funktionen}

\begin{karte}{Komplexe Differenzierbarkeit mit Cauchy-Riemann-DGL}
    Angenommen \(\abb{f}{U}{\C}\), \(U\) Gebiet in \(\C\), ist komplex 
    differenzierbar in \(z\in U\). Dann existieren die partiellen Ableitungen 
    \(f_x, f_y\) in \(z\) und erfüllen die Cauchy-Riemann-DGL 
    \[ f_y = i f_x \]
    in \(z\in U\) oder, schreibt man \(f=u+iv\) 
    \[ u_x = v_y, \quad u_y = -v_x. \]
    Alternativ fassen wir \(f\) als Abbildung von \(\R^2 \rightarrow \R^2\) auf. 
    \[ Df = (\partial_x f, \partial_y f) = \begin{pmatrix}
        \partial_x u & \partial_y u \\ \partial_x u & \partial_y u
    \end{pmatrix}. \]
    Die Umkehrung gilt i. A. nicht.
\end{karte}

\begin{karte}{Cauchy-Riemann-DGL mit komplexer Differenzierbarkeit}
    \( \abb{f}{U}{\C} \) besitze die partielle Ableitung \(f_x, f_y\) in einer Umgebung von \(z\) 
    und \(f_x, f_y\) seien stetig in dieser Umgebung und \(f_y = i f_x\). Dann ist \(f\) komplex 
    differenzierbar in \(z\).
\end{karte}

\begin{karte}{Analytisch}
    Eine Funktion \(f\) ist analytisch in \(z\) bzw. holomorph in \(z\), 
    falls \(f\) in einer Umgebung von \(z\) komplex differenzierbar ist.\\
    \(f\) ist analytisch in der Menge \(S\), falls \(f\) komplex differenzierbar 
    ist in einer offenen Menge, die \(S\) enthält. \\
    Ist \(\abb{f}{\C}{\C}\) auf \(\C\) analytisch, so heißt \(f\) ganze Funktion.
\end{karte}

\begin{karte}{Inverse}
    Angenommen \(S, T \subset \C\) sind offen, \(\abb{f}{S}{T}\) ist bijektiv. 
    \(g\) ist die Inverse von \(f\) auf \(T\), falls 
    \[ f(g(z)) = z \;\forall z\in T. \]
    \(g\) ist die Inverse von \(f\) in \(z_0\), falls \(g\) die Inverse von \(f\) 
    in einer Umgebung von \(z_0\) ist.
\end{karte}

\begin{karte}{Ableitung Inverse}
    Ist \(g\) die Inverse von \(f\) in \(z_0\) und ist \(g\) stetig in \(z_0\) 
    und \(f'(g(z_0)) \neq 0\), dann ist \(g\) differenzierbar in \(z_0\) und es gilt 
    \[ g'(z_0) = \frac{1}{f'(g(z_0))}. \]
\end{karte}

\begin{karte}{\(\Re(f)\), \(\abs{f}\) konstant}
    Angenommen \(f=u+iv\) ist analytisch in einem Gebiet \(D\) und \(u\) ist konstant. 
    Dann ist \(f\) konstant auf \(D\).\\
    Sei \(f\) analytisch auf einem Gebiet \(D\) und \(\abs{f}\) konstant, dann ist \(f\) konstant.
\end{karte}