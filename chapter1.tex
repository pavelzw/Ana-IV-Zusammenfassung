\section{Grundlagen}

\begin{karte}{Notationen 1}
    \begin{itemize}
        \item \(D_R(z_0) = U_R(z_0) = \set{z\in \C : \abs{z-z_0} < R}\) ist 
        die offene Kugel um \(z_0\) mit Radius \(R\).
        \item \(D_\varepsilon(z_0)\) wird auch \(\varepsilon\)-Umgebung von \(z_0\) genannt.
        \item \(C_R(z_0) = \partial D_R(z_0)\) Kreis um \(z_0\) mit Radius \(R\).
        \item \(S \subset C, S^C = \C \setminus S\) Komplement von \(S\).
        \item \(S\) ist abgeschlossen, falls \(S^C\) offen ist (bzw. \((z_n)_n \subset S, z_n \rightarrow z \Rightarrow z\in S\))
        \item \(S\) ist beschränkt, falls \(S \subset D_R(0)\) für ein \(R > 0\).
        \item \(S\) ist kompakt genau dann, wenn \(S\) abgeschlossen und beschränkt ist.
    \end{itemize}
\end{karte}

\begin{karte}{Notationen 2}
    \begin{itemize}
        \item \(S\) ist unzusammenhängend, falls offene Mengen \(A, B \subset \C\) existieren, \(A\cap B = \emptyset\) 
        mit \(S \subset A \cup B\) und \(S \not\subseteq A, S \not \subseteq B\).
        \item \(S\) ist zusammenhängend, falls \(S\) nicht unzusammenhängend ist.
        \item \([z_1, z_2] := \set{t z_1 + (1-t) z_2 \;|\; 0\leq t \leq 1}\).
        \item Polygonzug: endliche Vereinigung von zusammenhängenden Segmenten 
        der Form 
        \[ [z_0, z_1] \cup [z_1, z_2] \cup \cdots \cup [z_{n-1}, z_n]. \]
        \item Ist \(S\subset \C\) und können je zwei Punkte in \(S\) durch 
        einen Polygonzug verbunden werden, so heißt \(S\) 
        (polygonal) wegzusammenhängend. Jede polygonal wegzusammenhängende 
        Menge ist zusammenhängend, aber nicht umgekehrt.
    \end{itemize}
\end{karte}

\begin{karte}{Gebiet/Umgebung}
    Eine offene, zusammenhängende Menge heißt Gebiet oder 
    (offene) Umgebung.\\
    Jedes Gebiet ist polygonal zusammenhängend.
\end{karte}

\begin{karte}{\(M\)-Test}
    Angenommen \(f_n\) ist stetig auf \(D\) für alle \(n\in\N\), 
    ist \(\abs{f_n} \leq M_n\) auf \(D\) und \(sum_{n=1}^\infty M_n < \infty\), 
    so konvergiert die Reihe \(\sum_{n=1}^\infty f_n(x)\) (absolut und gleichmäßig auf \(D\)) 
    gegen eine stetige Funktion auf \(D\).
\end{karte}

\begin{karte}{Stereographische Projektion}
    Sei \(\Sigma := \set{ (\xi, \eta, \rho) \in \R^3 \;|\;\xi^2 + \eta^2 + (\rho - \frac{1}{2})^2 = \frac{1}{4} }\) Sphäre in \(\R^3\) 
    mit Radius \(\frac{1}{2}\) um den Mittelpunkt \((0,0,\frac{1}{2})\). 
    Ferner sei die Ebene \(\rho = 0\) identifiziert mit \(R^2 = \C\), \(\xi = x\)-Achse und \(\eta = y\)-Achse.\\
    Jeder Punkt \((\xi, \eta, \rho) \neq (0,0,1)\) wird mit dem Punkt \(z\in \C\) assoziiert, durch den der Strahl von \((0,0,1)\)
    durch \((\xi,\eta,\rho)\) die Ebene \(\R^2 = \C\) schneidet.

    Sei \(S\) ein Kreis in \(\Sigma\) und \(T\) die stereographische
    Projektion von \(S\) in \(\C\). Dann gilt 
    \begin{enumerate}
        \item Enthält \(S\) den Punkt \((0,0,1)\), so ist \(T\) eine Linie.
        \item Ist \((0,0,1) \notin S\), so ist \(T\) ein Kreis.
    \end{enumerate}
\end{karte}