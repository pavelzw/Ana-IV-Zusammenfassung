\section{Laurententwicklungen}

\begin{karte}{Laurentreihe Konvergenz}
    Seien \((\mu_n)_{n\in\Z} \subset \C\). Wir sagen \(\sum_{n=-\infty}^\infty \mu_n\) 
    existiert (konvergiert), falls die beiden Reihen \(\sum_{k=0}^\infty \mu_k\) 
    und \(\sum_{k=1}^\infty \mu_{-k}\) konvergieren. Wir setzen dann 
    \[ \sum_{n=-\infty}^\infty \mu_n := \sum_{k=1}^\infty \mu_{-k} + \sum_{k=0}^\infty \mu_k. \] 
    Sei \((a_n)_{n\in\Z} \subset \C\). Die Laurentreihe 
    \[ f(z) := \sum_{n=-\infty}^\infty a_n z^n \]
    konvergiert in der Menge \(D := \set{z\in \C: R_1 < \abs{z}<R_2}\) mit 
    \[ R_2 := \frac{1}{\limsup_{n\rightarrow \infty} \abs{a_n}^{1/n}} 
    \text{ und } R_1 := \limsup_{k\rightarrow\infty} \abs{a_{-k}}^{1/k}. \]
    Ist \(R_1 < R_2\), so ist \(D\) ein Annulus und \(f\) ist analytisch in \(D\).
\end{karte}

\begin{karte}{Laurententwicklung von \(f\) in Annulus}
    Ist \(f\) analytisch in dem Annulus \(A_{R_1, R_2} = \set{z\in\C: R_1 < \abs{z}<R_2}\), 
    so hat \(f\) eine konvergente Laurententwicklung 
    \[ f(z) = \sum_{n=-\infty}^\infty a_n z^n \text{ in } A_{R_1, R_2}. \]
    Es gilt 
    \[ a_k = \frac{1}{2\pi i} \int_{\partial C_r(0)} \frac{f(\zeta)}{\zeta^{k+1}} 
    \text{ für alle } R_1 < r < R_2. \]
    Die Laurententwicklung ist eindeutig.
\end{karte}

\begin{karte}{Darstellung Laurententwicklung}
    Ist \(f\) analytisch in dem Annulus \(A_{R_1, R_2}(z_0) = \set{z\in \C: R_1 < \abs{z-z_0} < R_2}\), 
    so hat \(f\) die eindeutige Darstellung 
    \[f(z) = \sum_{k=-\infty}^\infty a_k (z-z_0)^k, \]
    wobei \[ a_k = \frac{1}{2\pi i} \int_{\partial C_r(z_0)} \frac{f(z)}{(z-z_0)^{k+1}} \dx{z} \]
    und \(\partial C_r(z_0) = \set{z\in \C: \abs{z-z_0} = r}\) für beliebiges \(R_1 < r < R_2\).
\end{karte}

\begin{karte}{Laurententwicklung bei isolierten Singularitäten}
    Hat \(f\) in \(z_0\) eine isolierte Singularität und ist analytisch in 
    einer punktierten Umgebung von \(z_0\), so gilt für ein \(\delta > 0\) 
    und alle \(0<\abs{z-z_0}<\delta\) 
    \[ f(z) = \sum_{k=-\infty}^\infty a_k (z-z_0)^k. \]
\end{karte}

\begin{karte}{Hauptteil/analytischer Teil}
    Ist \(f(z) = \sum_{k\in\Z} a_k (z-z_0)^k\) die Laurententwicklung 
    einer Funktion \(f\) um die isolierte Singularität in \(z_0\), 
    so nennen wir \(\sum_{k=-\infty}^{-1}a_k (z-z_0)^k\) den Hauptteil 
    von \(f\) in \(z_0\) und \(\sum_{k=0}^\infty a_k (z-z_0)^k\) 
    den analytischen Teil von \(f\) in \(z_0\).\\
    \begin{itemize}
        \item Hat \(f\) eine hebbare Singularität in \(z_0\), so sind alle Koeffizienten 
        der Laurententwicklung \(a_k = 0\) für alle \(k \leq -1\).
        \item \(f\) hat einen Pol der Ordnung \(k\) in \(z_0\)
        \(\Leftrightarrow\) \(a_{-k} \neq 0\) und \(a_{-n} = 0\) für 
        alle \(n\geq k+1\).
        \item Hat \(f\) eine wesentliche Singularität in \(z_0\), 
        so hat die Laurententwicklung von \(f\) \(\infty\)-viele 
        Koeffizienten im Hauptteil \(\neq 0\), d. h. 
        \( \set{ k\in \Z: k<0 \text{ und } a_k \neq 0 } \) ist 
        unendlich groß.
    \end{itemize}
\end{karte}

\begin{karte}{Partialbruchzerlegung}
    Jede rationale Funktion der Form 
    \[ R(z) = \frac{P(z)}{G(z)} = \frac{P(z)}{(z - z_1)^{k_1} \cdots (z-z_n)^{k_n}}, \]
    wobei \(P\) und \(G\) Polynome vom Grad \(\deg P < \deg G\) sind, 
    kann durch eine Summe von Polynomen in den Variablen 
    \(\frac{1}{z-z_k}, k=1,\ldots, n\) dargestellt werden.
\end{karte}